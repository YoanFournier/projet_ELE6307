\documentclass[11pt,letterpaper]{article}

%\usepackage[latin1]{inputenc} 
\usepackage[utf8]{inputenc}
\usepackage[cyr]{aeguill}
\usepackage[colorlinks=true]{hyperref}
\usepackage{textpos}
\usepackage{graphicx}
\usepackage[french]{babel}
\usepackage{color}
\usepackage{array}
\usepackage{enumerate}
\usepackage{fancyhdr}
\usepackage{lastpage}
\usepackage{amsmath}
\usepackage{amssymb}
\usepackage{epstopdf}
\usepackage{mathrsfs}
\usepackage{float}
\usepackage{multicol}
\usepackage{caption}
\usepackage{subcaption}
\usepackage{siunitx}
\usepackage{rotating}
\usepackage{adjustbox}
\addto\captionsfrench{\def\tablename{Tableau}}

%-----------------------------------------------------------------%
% Definitions
\newcommand{\session}{Hiver 2022}
\newcommand{\firstauthor}{Yoan \textbf{Fournier}}
\newcommand{\firstregistrationnumber}{1958736}
\newcommand{\secondauthor}{Victor \textbf{Gaudreau-Blouin}}
\newcommand{\secondregistrationnumber}{1958297}
\newcommand{\reportnumber}{}
\newcommand{\firsttitle}{Analyse énergétique d'un processeur vectoriel pour des calculs de DNN}
\newcommand{\secondtitle}{Rapport final}
%-----------------------------------------------------------------%



\oddsidemargin 0pt
\topmargin 0pt
\textwidth 6.5in
\textheight 8.1in

\setlength{\parskip}{1em}

\definecolor{bleu_poly}{RGB}{65,170,230}
\definecolor{vert_poly}{RGB}{140,200,60}
\definecolor{orange_poly}{RGB}{250,150,30}
\definecolor{rouge_poly}{RGB}{185,30,50}
\definecolor{gris_poly}{RGB}{166,168,171}


\title{\vspace{-2.5cm} \noindent\makebox[\linewidth]{\color{rouge_poly}{\rule{\textwidth}{1.5pt}}}
        \begin{center}
        \begin{tabular}{m{6.5cm}m{6cm}}
        \textbf{ \huge Projet \reportnumber}  & \includegraphics[width=0.4\textwidth]{Polytechnique_signature-CMYK-droite_FR.eps}
        \end{tabular}
        \end{center}
        \noindent\makebox[\linewidth]{\color{rouge_poly}{\rule{\textwidth}{1.5pt}}}
        \\ \  \\
        \Huge \firsttitle \\ \secondtitle  
        \\ \ \\
        \LARGE ELE6307 - Machines neuronales : architectures et applications
        }

\author{\session \\ Département de génie électrique \\ École Polytechnique de Montréal} 

\date{Dernière mise à jour: \today}

\pagestyle{fancy}

\lfoot{\session}
\cfoot{ELE6307 - Machines neuronales : architectures et applications}
\rfoot{\thepage/\pageref{LastPage}}
\chead{}
\lhead{\emph{Projet -- \firstauthor  \, (\firstregistrationnumber)/\secondauthor\,  (\secondregistrationnumber)}}
\rhead{\includegraphics[width=2.5cm]{Polytechnique_signature-CMYK-droite_FR.eps}}
\renewcommand{\headrulewidth}{0.4pt}
\renewcommand{\footrulewidth}{0.4pt} 
\setlength{\headheight}{45pt}


\graphicspath{{fig/}}


\newcommand{\vb}[1]{\mathbf{#1}}
\newcommand{\bs}[1]{\boldsymbol{#1}}


%-----------------------------------------------------------------%
% SOF
\begin{document}
\maketitle
\noindent\makebox[\linewidth]{\color{rouge_poly}{\rule{\textwidth}{1.5pt}}} 


\noindent \LARGE \firstauthor  \hfill \firstregistrationnumber


\noindent \LARGE \secondauthor \hfill \secondregistrationnumber


\noindent\makebox[\linewidth]{\color{rouge_poly}{\rule{\textwidth}{1.5pt}}}


\newpage
\normalsize

\section{Introduction}
    \begin{multicols}{2}
    Dans les dernières années, l'utilisation de réseaux de neurones profonds (DNN) pour
    résoudre différentes tâches a beaucoup augmenté. Ces DNNs nécessitent des puissances
    de calculs considérables, mais à la fois nécessitent une bonne efficacité énergétique 
    étant donné leur déploiement dans des appareils mobiles. Bien que les processeurs 
    généralistes que l'on retrouve aujourd'hui ont augmenté rapidement en performance,
    les limitations du \textit{Dennard Scaling} se font ressentir. Il est donc souhaitable
    de trouver une autre approche pour accélérer de façon efficace les calculs dans les DNNs.
    Une manière intéressante pour l'accélération des calculs pour des DNNs est l'utilisation
    de processeurs vectoriels plutôt que des processeurs scalaires standards. Ces processeurs
    utilisent un jeu d'instruction SIMD (\textit{Single instruction, Multiple Data}) plutôt 
    que SISD (\textit{Single instruction, Single data}). Ainsi une instruction SIMD peut
    performer la même opération sur plusieurs données plutôt qu'une seule, ce qui permet 
    de plus facilement augmenter le parallélisme des calculs. 
    
    Le présent rapport étudie l'efficacité énergétique d'un modèle simplifié du processeur
    vectoriel ARA sur différents problèmes en utilisant le simulateur \textit{Timeloop-Accelergy}.
    \end{multicols}

\section{Méthodologie}
    \begin{multicols}{2}

    L'architecture sur laquelle notre modèle se base est sur celle du coprocesseur 
    vectoriel ARA \cite{ara_paper}. Ce coprocesseur est une extension vectorielle du processeur scalaire
    RISCV CVA6. Le fonctionnement d’ARA se base sur le concept de \textit{Lanes}. 
    Les figures \ref{fig:top_arch} et \ref{fig:lane_arch} présentent l'architecture d’ARA \cite{bougenot_2020}.
    Le coprocesseur ARA possède un nombre d'allées, ou \textit{Lanes}, qui sont capables de traiter des
    éléments vectoriels indépendamment. Le routage des données vers ces \textit{Lanes} grandit évidemment en complexité
    avec le nombre de \textit{Lanes}. Le \textit{Sequencer} s'occupe de lire les instructions à exécuter 
    et transmettre les informations correctes au \textit{Vector Load and Store Unit} (VLSU), 
    \textit{Slide Unit} (SLDU) ou aux \textit{Lanes}. Le VLSU est responsable de générer les bonnes
    adresses, amenant les données séquentiellement vers et depuis les \textit{Lanes}. 
    Le SLDU exécute des opérations qui n'ont pas d'indépendance entre les \textit{Lanes}.
    Finalement, chaque \textit{Lane} a son propre \textit{sequencer} pour les instructions à exécuter
    sur son élément vectoriel. Le reste de la \textit{Lane} peut être considéré comme un PE, avec sa mémoire locale,
    un étage de conversion et un étage d'exécution.

    {\centering
    \includegraphics[width=\linewidth]{ara_arch.png}
    \captionsetup{hypcap=false}
    \captionof{figure}{Architecture globale}
    \label{fig:top_arch}}
    \bigskip

    Afin de modéliser l'efficacité énergétique du coprocesseur vectoriel ARA, le simulateur Timeloop a été 
    utilisé. Ce simulateur utilise une structure hiérarchique pour modéliser une architecture. Une structure
    détaillant le problème à résoudre par l'architecture est aussi spécifiée. La commande \textit{timeloop-mapper}
    permet ensuite de trouver automatiquement la correspondance idéale du problème sur l'architecture matérielle.

    {\centering
    \includegraphics[width=\linewidth]{lane_arch.png}
    \captionsetup{hypcap=false}
    \captionof{figure}{Architecture \textit{Lane}}
    \label{fig:lane_arch}}
    \bigskip

    L'architecture simplifiée d'ARA, comme mentionnée dans la proposition du projet, peut se modéliser en termes de 
    \textit{Lanes}. Chaque \textit{Lanes} à sa plus simple expression consiste en une banque de 8 registres ainsi qu'une
    unité arithmétique capable d'effectuer des multiplications et des additions. Bien que sur ARA, l'unité d'addition 
    est séparée de l'unité de multiplication, les deux unités peuvent fonctionner durant le même cycle sur des données
    indépendantes. Dans le cadre de la modélisation, on simplifie en utilisant la structure \textit{intmac} fournie par
    l'outil Timeloop-Accelergy. Nous considérons que la différence causée par cette simplification est minimale. En effet,
    les opérations dans une Lane sont pipelinées et les opérations de multiplication et addition peuvent être exécutées en même temps.
    La seule différence principale est la latence causée par le pipelinage des opérations. Cependant, ce n'est pas une métrique
    de performance étudiée dans le cadre de ce projet.

    {\centering
    \includegraphics[width=0.8\linewidth]{arch_visio.png}
    \captionsetup{hypcap=false}
    \captionof{figure}{Architecture de ARA sur Timeloop}
    \label{fig:timeloop_arch}}
    \bigskip

    L'architecture utilisée est présentée à la figure \ref{fig:timeloop_arch}. On y observe une mémoire DRAM principale reliée à 
    la mémoire cache L1 de 64 KB. La cache L1 est reliée à un nombre paramétrable de \textit{Lanes}. 
    Chaque \textit{Lane} comporte sa banque de 8 registres et son unité de \textit{multiply-accumulate}.

    Afin de trouver les meilleurs mappings, la commande \textit{timeloop-mapper} a été utilisée. 
    L'optimisation du mapping se fait en utilisant comme seule métrique l'énergie. L'ensemble des résultats
    ont étés obtenues en utilisant la technologie 45 nm fournie par Accelergy. Aussi, un problème sur le 
    \textit{victory-condition} du mapping a été trouvé en utilisant \textit{timeloop-mapper}. Ainsi, le code
    source de Timeloop a été modifié afin d'obtenir des résultats. Le problème est relié à un \textit{flag} 
    qui n'était pas correctement configuré. Ainsi, le compteur du \textit{victory-condition} ne s'incrémentait
    pas correctement.

    Afin d'étudier la performance de l'architecture en fonction du nombre de \textit{Lanes}, les problèmes utilisés sont 
    les trois premières couches convolutionnelles de VGG16 et les trois premières couches convolutionnelles d'AlexNet. 
    Ces deux groupes de problèmes ont été choisis pour observer la variation de la taille du problème, du nombre de canaux, 
    le nombre de filtres et la taille des \textit{kernels}. Avec VGG16, chaque couche convolutionnelle possède la même 
    taille de \textit{kernel}, mais varie en canaux, filtres et taille. Pour AlexNet, nous avons trois différentes tailles de \textit{kernels}.
    La figure \ref{fig:prob} illustre les dimensions des problèmes étudiés.
    
    \begin{figure}[H]
        \centering
        %\includegraphics[width=\linewidth]{probs.png}
        \caption{Problèmes étudiés}
        \label{fig:probs}
    \end{figure}

    Afin de mesurer la performance énergétique, la métrique utilisée est le GFLOP/J. La 
    formule utilisée pour calculer cette métrique est la suivante :
    $$\text{GFLOP/J} = \frac{\%utilization \times \#cycles \times \#lanes}{E_{tot}}$$

    L'ensemble des fichiers sources utilisés afin d'obtenir les résultats sont disponibles
    sur le repo GitHub suivant : \url{https://github.com/YoanFournier/projet_ELE6307}.

    \end{multicols}

\section{Résultats}
    \begin{multicols}{2}

    Cette section présente les résultats d'aire, de cycles, d'énergie et de performance en fonction
    du nombre de \textit{Lanes} pour les 3 premières couches de AlexNet et VGG16.

    Tout d'abord, la figure \ref{fig:area} montre que l'aire du circuit augment de façon linéaire en fonction
    du nombre de \textit{Lanes}. C'est un résultat attendu puisque chaque \textit{Lane} prend, à priori, la même surface.
    Il reste important de comprendre que ce résultat est approximatif, en réalité, la complexité du P\&R risque d'augmenter
    plus rapidement que de façon linéaire en fonction du nombre de \textit{Lanes}.

    Les figures \ref{fig:cycles_alex} et \ref{fig:cycles_vgg} montrent la corrélation entre le nombre de cycles 
    pour effectuer le calcul des couches convolutive en fonction du nombre de \textit{Lanes}. Bien que l'on pourrait
    s'attendre à une réduction linéaire des cycles, on remarque que ce n'est pas tout à fait le cas. En effet, l'utilisation
    des \textit{Lanes} n'est pas de 100\% pour tous les problèmes comme le démontre la rangée \textit{Utilization} des 
    tableaux \ref{tab:results_alex} et \ref{tab:results_vgg}. Le mapper de Timeloop n'arrive donc pas toujours à trouver un 
    mapping utilisant plus de ressources qui optimise la consommation d'énergie.

    {\centering
    \includegraphics[width=\linewidth]{Alex_cycles.png}
    \captionsetup{hypcap=false}
    \captionof{figure}{Cycles de ARA en fonction du nombre de \textit{Lanes} pour le problème AlexNet}
    \label{fig:cycles_alex}}
    \bigskip

    {\centering
    \includegraphics[width=\linewidth]{VGG_cycles.png}
    \captionsetup{hypcap=false}
    \captionof{figure}{Cycles de ARA en fonction du nombre de \textit{Lanes} pour le problème VGG16}
    \label{fig:cycles_vgg}}
    \bigskip

    Les figures \ref{fig:energy_alex} et \ref{fig:energy_vgg} présentent l'énergie consommée pour les différents calculs
    en fonction du nombre de \textit{Lanes}. On remarque une tendance à la baisse dans la consommation d'énergie plus le nombre 
    de \textit{Lanes} est élevé. En effet, plus il y a de \textit{Lanes}, plus il y a de banques de registres disponibles. Ainsi,
    la quantité de mémoire locale aux unités de calcul est plus élevée et il est moins nécessaire d'avoir accès aux hiérarchies de 
    mémoires plus élevées comme la cache ou la DRAM. Ceci à pour effet de réduire la consommation d'énergie. Il reste que ces gains
    énergétiques sont assez faibles et dans l'ensemble le meilleur gain d'énergie observé est de 1.52\% pour la couche 2 de VGG16 entre 
    une \textit{Lane} et 16 \textit{Lanes}. 

    {\centering
    \includegraphics[width=\linewidth]{Alex_energy.png}
    \captionsetup{hypcap=false}
    \captionof{figure}{Énergie de ARA en fonction du nombre de \textit{Lanes} pour le problème AlexNet}
    \label{fig:energy_alex}}
    \bigskip

    {\centering
    \includegraphics[width=\linewidth]{VGG_energy.png}
    \captionsetup{hypcap=false}
    \captionof{figure}{Énergie de ARA en fonction du nombre de \textit{Lanes} pour le problème VGG16}
    \label{fig:energy_vgg}}
    \bigskip

    Les figures \ref{fig:perf_alex} et \ref{fig:perf_vgg} mettent en avant l'efficacité énergétique en GFLOP/J en fonction du 
    nombre de \textit{Lanes}. Le même constat que pour la consommation énergétique peut-être observée. Il y a une tendance à la
    hausse de l'efficacité, mais les gains sont minimes et restent plutôt constants. On observe au maximum, un gain de 1.92\% entre une
    \textit{Lane} et 16 \textit{Lanes} pour la couche 1 de AlexNet.

    {\centering
    \includegraphics[width=\linewidth]{Alex_performance.png}
    \captionsetup{hypcap=false}
    \captionof{figure}{Performance de ARA en fonction du nombre de \textit{Lanes} pour le problème AlexNet}
    \label{fig:perf_alex}}
    \bigskip

    {\centering
    \includegraphics[width=\linewidth]{VGG_performance.png}
    \captionsetup{hypcap=false}
    \captionof{figure}{Performance de ARA en fonction du nombre de \textit{Lanes} pour le problème VGG16}
    \label{fig:perf_vgg}}
    \bigskip

    À la lumière de ces résultats, un aspect intéressant peut être soulevé. On remarque qu'en général, le mapper de 
    Timeloop semble préférer effectuer le mapping spatial sur la composante C du problème, soit le nombre de canaux d'entrée.
    La rangée \textit{Utilization} du tableau \ref{tab:results_alex} le démontre bien. On remarque une utilisation de 
    75 \%, 75 \% et 94 \% pour 4, 8 et 16 \textit{Lanes} respectivement. Ceci correspond à une utilisation de 3, 6 et 15 \textit{Lanes}
    qui sont tous des multiples de C=3. L'avantage de faire la distribution spatiale des données selon C est efficace énergétiquement
    puisque l'ensemble des canaux sont indépendants. Chaque \textit{Lane} peut donc effectuer ses calculs de façon indépendante des autres
    ce qui permet de limiter les transferts mémoire entre les \textit{Lanes} et à comme effet de minimiser la consommation énergétique.
    Comme la plupart des couches internes d'un DNN possèdent un nombre pair de canaux d'entrée, il serait sans doute intéressant d'utiliser
    un mapping spatial sur C dans la majorité des cas. Cependant, on remarque dans les tableaux \ref{tab:results_alex} et \ref{tab:results_vgg} 
    que l'utilisation n'est pas de 100 \% pour les problèmes ayant un C pair.  

    Afin de tester l'hypothèse qu'un mapping spatial sur C est généralement plus intéressant, un mapping custom a été 
    conçu afin de le comparer au mapping effectué par Timeloop. Le problème utilisé est la couche 2 d'AlexNet sur 16 \textit{Lanes}.
    Les deux mappings sont présentés au figures \ref{fig:map_timeloop_alex_L2} et \ref{fig:map_custom_alex_L2}.
    La couche 2 d'AlexNet comporte 96 canaux, ce qui est divisible par le nombre de \textit{Lanes}. Cependant, le mapping 
    de Timeloop n'utilise que 62 \% des \textit{Lanes} comme le montre le tableau \ref{tab:alex_L2_custom_map}. En utilisant 
    un mapping custom faisant le mapping spatial sur C, on obtient une utilisation de 100 \% ainsi qu'un gain énergétique
    et un gain en efficacité. 

    \begin{table}[H]
        \centering
        \begin{tabular}{l|cc}
        Problem                     & \multicolumn{2}{c}{\textbf{AlexNet\_layer2}}                                           \\
        Parameters                  & \multicolumn{2}{c}{\begin{tabular}[c]{@{}c@{}}M=256 K=5x5\\ C=96 I=27x27\end{tabular}} \\
        n                           & \multicolumn{2}{c}{16}                                                                 \\
        Mapper                      & \multicolumn{1}{c|}{Timeloop}                         & Custom                         \\ \hline
        Utilization                 & 0.62                                                  & 1                              \\
        Cycles {[}Millions{]}       & 44.8                                                  & 28.0                           \\
        Energy {[}mJ{]}             & 15.92                                                 & 15.90                          \\
        \textbf{Perf {[}GFLOP/J{]}} & 27.90                                                 & 28.18                         
        \end{tabular}
        \caption{Comparaison entre un mapping Timeloop et un mapping custom pour AlexNet L2}
        \label{tab:alex_L2_custom_map}
    \end{table}    
    \bigskip

    En général, les résultats démontrent clairement l'avantage que peut apporter un processeur vectoriel. En effet,
    bien que la consommation et l'efficacité énergétique restent similaires pour les différents nombres de \textit{Lanes},
    il faut comprendre qu'augmenter le nombre de \textit{Lanes} réduit le nombre de cycles nécessaires afin d'effectuer 
    un calcul. En effet, sil’on utilise 100 \% des ressources, l'utilisation de 16 \textit{Lanes} apporte un gain en performance
    jusqu'à un facteur 16 par rapport à un processeur scalaire et ceci pour une efficacité énergétique identique, voire meilleure.

    {\centering
    \includegraphics[width=\linewidth]{area.png}
    \captionsetup{hypcap=false}
    \captionof{figure}{Aire de ARA en fonction du nombre de \textit{Lanes}}
    \label{fig:area}}
    \bigskip

    \end{multicols}

\section{Conclusion}
    \begin{multicols}{2}
    %TODO: Retour sur resultats. Plus de Lanes = meilleure perf pour mm energie.
    %TODO: Retour sur l'atteinte des objectifs de la proposition
    %TODO: Optim avec GFLOP/J directement + contraindre a faire mapping spatial sur C
    %TODO: Modeliser SLDU + sequencer
    
    \end{multicols}

{\bibliography{refs}}
    \bibliographystyle{abbrv}

\pagebreak
\section*{Annexe}

    \begin{figure}[H]
        \centering
        \includegraphics[width=0.9\linewidth]{map_timeloop_alex_L2.png}
        \caption{Mapping Timeloop pour AlexNet L2}
        \label{fig:map_timeloop_alex_L2}
    \end{figure}

    \begin{figure}[H]
        \centering
        \includegraphics[width=0.9\linewidth]{map_custom_alex_L2.png}
        \caption{Mapping custom pour AlexNet L2}
        \label{fig:map_custom_alex_L2}
    \end{figure}

    \begin{sidewaystable}[h]
    \vspace{-6cm}
        \footnotesize
        \begin{tabular}{l|ccccc|ccccc|ccccc}
        Problem                     & \multicolumn{5}{c|}{\textbf{AlexNet\_layer1}}                                           & \multicolumn{5}{c|}{\textbf{AlexNet\_layer2}}                                          & \multicolumn{5}{c}{\textbf{AlexNet\_layer3}}                                            \\
        Parameters                  & \multicolumn{5}{c|}{\begin{tabular}[c]{@{}c@{}}M=96 K=11x11\\ C=3 I=55x55\end{tabular}} & \multicolumn{5}{c|}{\begin{tabular}[c]{@{}c@{}}M=256 K=5x5\\ C=96 I=27x27\end{tabular}} & \multicolumn{5}{c}{\begin{tabular}[c]{@{}c@{}}M=384 K=3x3\\ C=256 I=13x13\end{tabular}} \\
        n                           & 1                & 2                & 4               & 8              & 16             & 1                & 2               & 4               & 8              & 16             & 1                & 2                & 4               & 8              & 16             \\ \hline
        Utilization                 & 1                & 1                & 0.75            & 0.75           & 0.94           & 1                & 1               & 0.75            & 1              & 0.62           & 1                & 1                & 0.75            & 1              & 1              \\
        Cycles {[}Millions{]}       & 105.4            & 52.7             & 35.1            & 17.6           & 7.3            & 447.9            & 223.9           & 149.3           & 56.0           & 44.8           & 149.5            & 74.8             & 49.8            & 18.7           & 9.3            \\
        Energy {[}mJ{]}             & 3.82             & 3.80             & 3.77            & 3.77           & 3.76           & 16.03            & 15.99           & 15.98           & 15.95          & 15.92          & 5.40             & 5.39             & 5.36            & 5.34           & 5.35           \\
        \textbf{Perf {[}GFLOP/J{]}} & 27.59            & 27.73            & 27.91           & 28.00          & 28.12          & 27.94            & 28.01           & 28.02           & 28.08          & 27.90          & 27.69            & 27.73            & 27.86           & 28.00          & 27.92         
        \end{tabular}
        \caption{Résultats pour le problème AlexNet}
        \label{tab:results_alex}
        
    \bigskip
    \bigskip
    \bigskip
    \bigskip

        \footnotesize
        \begin{tabular}{l|ccccc|ccccc|ccccc}[
        Problem                     & \multicolumn{5}{c|}{\textbf{VGG02\_layer1}}                                             & \multicolumn{5}{c|}{\textbf{VGG02\_layer2}}                                              & \multicolumn{5}{c}{\textbf{VGG02\_layer3}}                                               \\
        Parameters                  & \multicolumn{5}{c|}{\begin{tabular}[c]{@{}c@{}}M=64 K=3x3\\ C=3 I=224x224\end{tabular}} & \multicolumn{5}{c|}{\begin{tabular}[c]{@{}c@{}}M=64 K=3x3\\ C=64 I=224x224\end{tabular}} & \multicolumn{5}{c}{\begin{tabular}[c]{@{}c@{}}M=128 K=3x3\\ C=64 I=112x112\end{tabular}} \\
        n                           & 1                & 2               & 4               & 8               & 16             & 1                & 2               & 4               & 8               & 16              & 1                & 2               & 4               & 8               & 16              \\ \hline
        Utilization                 & 1                & 1               & 0.75            & 0.75            & 0.75           & 1                & 1               & 1               & 1               & 1               & 1                & 1               & 1               & 0.5             & 1               \\
        Cycles {[}Millions{]}       & 86.7             & 43.4            & 28.9            & 14.5            & 7.2            & 1850             & 924.8           & 462.4           & 231.2           & 115.6           & 925              & 462             & 231             & 231             & 57.8            \\
        Energy {[}mJ{]}             & 3.31             & 3.30            & 3.27            & 3.27            & 3.26           & 66.9             & 66.7            & 65.9            & 65.9            & 65.7            & 33.2             & 33.2            & 32.9            & 32.9            & 32.9            \\
        \textbf{Perf {[}GFLOP/J{]}} & 26.18            & 26.28           & 26.48           & 26.55           & 26.59          & 27.65            & 27.72           & 28.08           & 28.07           & 28.15           & 27.82            & 27.84           & 28.13           & 28.07           & 28.13          
        \end{tabular}
        \caption{Résultats pour le problème VGG16}
        \label{tab:results_vgg}  
    \end{sidewaystable}

\end{document} 
%-----------------------------------------------------------------%
% EOF